
% snowdon-lesson-02.tex - Our first LaTeX example!

\documentclass{geocraft-worksheet}

\usepackage[pdftex,
             pdfauthor={Sarah Zaman \& Dave Ames},
            pdftitle={GeoCraft: Programming on Snowdon Part 2 -
              Iteration}, 
            pdfcreator={LaTeX with hyperref and listings}]{hyperref}


\begin{document}


\title{Programming on Snowdon}
\subtitle{Part 2 - Iteration}

\maketitle

\pagenumbering{gobble}

\setcounter {section} {2}


\section{Let it Snow}
\lstset{language=Python}

\noindent% Line above MUST be blank
\tikzmark{Start}%
\lstinputlisting{03-snow.py}\vspace{0.2cm}
\tikzmark{End}
\AddBackgroundImage{Start}{End}{snowdon}%
%

\begin{itemize}
\item\textbf{Bronze Challenge:} Change the block type to a different
  block. (See sheet for choices). To make it exploding TNT use \lstinline{block.TNT,id, 1}. 
\item\textbf{Silver Challenge:} Change the coordinates \textbf{x, y}
    and \textbf{z} to leave a trail in different places. Try above your head. 
\item\textbf{Gold Challenge:} Place more than one block and leave a
  trail of \lstinline{GLOWING_OBSIDIAN} and \lstinline{GOLD_ORE}.
\end{itemize}

\section{Random Trails}

\noindent% Line above MUST be blank
\tikzmark{Start}%
\lstinputlisting{04-randomish.py}\vspace{0.2cm}
\tikzmark{End}
\AddBackgroundImage{Start}{End}{snowdon}%
%

\begin{itemize}
\item\textbf{Bronze Challenge:} Change the code to choose a random
  type of \lstinline{STONE_SLAB}.
\item\textbf{Silver Challenge:} Lay multicoloured paths round Mount Snowdon using \lstinline{block.WOOL.id}.
\item\textbf{Gold Challenge:} Use
  \lstinline{random.choice([block.STONE.id, block.DIRT.id])}, but try
  a variety of different blocks.

\end{itemize}



\end{document}

%%% Local Variables: 
%%% coding: utf-8
%%% mode: latex
%%% TeX-master: t
%%% End: 
