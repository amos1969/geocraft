
\documentclass{geocraft-worksheet}

\usepackage[pdftex,
            pdfauthor={Sarah Zaman \& Dave Ames},
            pdftitle={GeoCraft: Overview of Files},
            pdfcreator={LaTeX with hyperref and listings}]{hyperref}

\begin{document}

\title{Overview}
\subtitle{Of Files Contained in the Download}

\date{}

\maketitle

\section*{General Files}
\begin{itemize}
\item Overview - This document
\item Quick Start Guide For Teachers - If you have a reasonably up to
  date version of the Raspberry Pi Operating System, try using these
  instructions to get started first.
\item Getting Started Guide For Teachers - A more detailed guide to
  getting things set up on the Raspberry Pi to enable you to talk to
  Minecraft from Python.
\item Getting Started Guide For Students - A handout which can be
  given to Students to get them started with using and programming
  Minecraft on the Raspberry Pi.
\item Minecraft Blocks - A list of all of the blocks which can be
  accessed on Minecraft on the Raspberry Pi.
\end{itemize}

\section*{Maps Folder}
\begin{itemize}
\item Snowdon Folder - This contains the map of Snowdon for Minecraft
  to use (this whole folder needs to be copied to the right place. 
\item ResetGeoCraftMaps.py - A Python program which will copy the maps
  folders from this folder to the correct place on the Raspberry Pi,
  automatically. 
\item geocraftlogo - An image containing the GeoCraft Logo
\end{itemize}

\section*{Lesson Folder}
\begin{itemize}
\item Lesson Plan 1 - The lesson plan describing the first lesson.
\item Worksheet 1 - The worksheet which contains the details of the
  lesson for students to use.
\item Lesson Plan 2 - The lesson plan describing the second lesson.
\item Worksheet 2 - The worksheet which contains the details of the
  lesson for students to use.
\end{itemize}


\end{document}

%%% Local Variables:
%%% mode: latex
%%% TeX-master: t
%%% End:
