
\documentclass{geocraft-lesson-plan}

\usepackage[pdftex,
            pdfauthor={Sarah Zaman \& Dave Ames},
            pdftitle={GeoCraft Lesson Plan: Programming Snowdon Part 2- Iteration},
            pdfcreator={LaTeX with hyperref and listings},
            urlbordercolor={1 1 1}]{hyperref}


\begin{document}

\mytitle{Programming Snowdon}
\subtitle{Part 2 - Iteration}
%\agerange{5/6/7}
\duration{1 hour}

\section*{Overview} This is one of a series of lessons aimed at using Minecraft and Python on the Raspberry Pi to
develop students’ basic understanding of programming. In this series we use the area around Mount Snowdon as our
location, and build up gradually to the point where students will have enough knowledge to make changes to the
environment on Snowdon.  

\section*{Objectives}
\begin{itemize}
\item Use sequencing in programs.
\item Designs simple algorithms using loops.
\item Uses loops within a program
\item Detect and correct errors in algorithms and programs.
\item Use logical reasoning to explain how a simple algorithm works. 
\item Design and debug programs that accomplish certain goals.
\end{itemize}

\section*{Python constructs used}
\begin{itemize}
\item Use an external code library.
\item Create a variable and assign the return value from a function to it.
\item Use a while loop (actually an infinite loop).
\item Use an externally defined function.
\item Pass parameters to an externally defined function.
\item Create a list.
\item Select an item randomly from a list.
\end{itemize}

\section*{Minecraft API Functions Used}
\begin{itemize}
\item Import the main Minecraft module.
\item Import the blocks module..
\item Call the crate function to establish a link to Minecraft.
\item Get the player's position and store it in a variable.
\item Call the setBlock() function in order to place a block below the player's feet.
\item Pass data to a block.
\item Place multiple blocks by using multiple setBlock() functons.
\end{itemize}

\section*{Video Starter}
\begin{itemize}
\item YouTube explanation of iteration (by Mark Zuckerberg):
 {\textcolor{greenish} 
  {\url{ https://m.youtube.com/watch?v=mgooqyWMTxk}}} 
\end{itemize}

\section*{Unplugged Starter}
\begin{itemize}
\item Use this resource from code.org, to help students grasp the concept of iteration (loops):  
  {\textcolor{greenish}
    {\url{https://code.org/curriculum/course2/5/Activity5-GettingLoopy.pdf}}} 
\item If you feel that the students need to spend longer on loops and iteration, there is a complete lesson here:  
  {\textcolor{greenish}
    {\url{https://code.org/curriculum/course2/5/Teacher}}} 
\end{itemize}

\section*{Programming Minecraft Reminder}
\begin{itemize}
\item Remind students how to move in the game if necessary.
\item Remind them to start the Python Shell, using IDLE3, in the Programming Section of the Raspberry Pi Main Menu.
\item Remind them to create a new file to hold their program and save it with an appropriate name.
\item Also remind them that the Shell is not where the program is written, but is where any error messages will be
  displayed. 
\item Remind them that when they make a change they will need to rerun their program to ensure it works.
\end{itemize}

\section*{Related Activities}
\begin{itemize}
\item Use Lego blocks, multilink or other cubes to allow students to plan out their constructions.
\item Squared paper may help students visualise how the blocks are arranged.
\item Discuss visualisations of 3D shapes in Maths
\item Perhaps link it directly to some Maths activities such as: 
  {\textcolor{greenish}
    {\url{https://nrich.maths.org/2392}}}
\end{itemize}

\section*{Main Activity}
\begin{itemize}
\item Remind students that they will be entering Python code to create a program which will make things happen in the
  Minecraft world.
\item Explain to students that they will be entering code that will place a trail of blocks beneath their feet (and
  elsewhere) wherever they go.
\item The first piece of code that they will enter does the following.
\item The first line gives us access to the Minecraft Python library.
\item The second imports the details of all of the blocks which are available to us in Minecraft.
\item The third line creates the connection to Minecraft using the library we imported in the very first line and then
  stores it in the variable \textbf{mc}, which is what we then use, whenever we need to communicate with the game.
\item Line five, introduces a new Python construct, namely the \textbf{while} loop. This is the basis of the iteration
  we use in this activity. The command \textbf{while True:} starts an infinite loop (one that won't stop unless we tell
  it to). Each of the commands, which are indented, on the following lines will be executed in order, over and over, until
  we stop the program from running.
\item The sixth line, which gets executed on each iteration of the loop (each time we go through it), uses the link to
  Minecraft (\textbf{mc}) to get the player's position (as 3d coordinates x, y and z), and stores them in the variable
  \textbf{pos}.
\item The seventh line of code, uses our connection to Minecraft once more, this time to set a block in the world. The
  \textbf{setBlock} function takes a number of parameters, the first three are the coordinates at which to place the
  block. The second one \textbf{pos.y-1} specifies that the block we place will be beneath the players feet
  (\textbf{pos.y} would have placed the block where the bottom half of the player's body is situated). The fourth
  parameter is the type of block we wish to place (here it is a snow block). There is also a fifth optional parameter
  which would be data to be passed to the block to change it's behaviour, we don't use this here.
\item These last two lines are repeated over and over, placing a block of snow beneath the player's feet.
\item The \textbf{Bronze Challenge} then gets us to change the block type which we are placing beneath our feet, the
  \textbf{Minecraft Blocks} sheet has a list of the available blocks. Change the \textbf{block.SNOW.id} code replacing
  the \textbf{SNOW} part with the name of the block we want (in capitals). If you choose to place exploding TNT, then
  we supply the optional fifth parameter to the \textbf{setBlock} function. That is we put \textbf{, 1} after the
  \textbf{block.TNT.id} code and before the closing bracket.
\item The \textbf{Silver Challenge} then asks the student to experiment with modifying the coordinates \textbf{pos.x,
    pos.y} and \textbf{pos.z} in order to vary where the block appears relative to the player. This is best done by
  adding or subtracting a value to one of these variables, so that the block is placed relative to the player. If they
  only replace them with a number, then although the blocks will still be placed, it may be difficult to track down
  where the blocks have been placed.
\item The \textbf{Gold Challenge} asks the students to make multiple trails appear relative to the player. This can be
  done by duplicating the final line of code (the one with the \textbf{setBlock} function). Make sure it is indented in
  line with the other lines. Change at least one coordinate in this last line of code (if we keep the coordinates
  identical then the block placed by this final line of code, will replace the block placed in the previous one), and
  possibly the block type too. 
\item The second half of the worksheet, details how to make random blocks (and types of blocks) appear beneath the
  players feet.
\item The example code starts of in the same way as the previous code, the first change is on line three, where we have
  a new piece of code, which gives us access to the \textbf{random} library which has a variety of functions installed,
  that deal with randomness (or at least pseudo-randomness) in Python.
\item The next new line of code is at line eight which creates a new variable called \lstinline{block_type} and sets it to
  a random integer (whole number) between 0 and 3, inclusive. This is reset to a new value between 0 and 3 on each
  iteration of the loop (each time the loop runs).
\item This time the last line of code, sets the block to be a \lstinline{STONE_BRICK} block, but passes it the random value
  that is stored in \lstinline{block_type} as data. The \lstinline{STONE_BRICK} block type comes in 4 different versions,
  stone brick, mossy stone brick, cracked stone brick and chiseled stone brick. These correspond to the random value
  stored in \lstinline{block_type}. So each time through the loop, the block under the player's feet is replaced by one of
  the four \lstinline{STONE_BRICK} types. If the player stands still, and looks at their feet, then it will be possible to
  see the block flickering as the block is constantly replaced. 
\item The \textbf{Bronze Challenge} asks the students to change the block to be a \lstinline{STONE_SLAB}, if they set line
  eight to choose random values between 0 and 5 then this time it will choose one of the six types of block which a
  \lstinline{STONE_SLAB} can take.
\item The \textbf{Silver Challenge} asks the students to change the block which is being placed to a \textbf{WOOL}
  block. If they then change the random value to be one between 0 and 15 it will this time use all 16 different possible
  colours to make a random multicoloured trail behind the player.
\item The \textbf{Gold Challenge} changes the function we use to make a random choice. This time make line eight be the
  line of code shown on the sheet as an example (the one beginning \lstinline{some_blocks}. This line of code creates a
  list, which contains various blocks. They can add more blocks to the list, separating them from previous ones using
  commas. If they then change the final line to read: 
  \lstinline{mc.setBlock(pos.x, pos.y-1, pos.z,random.choice(some_blocks))} 
  then it will randomly select a random block from the list, each time through the loop, to be placed under the player's
  feet. 
\end{itemize}

\section*{Plenary - Suggested Activities}
\begin{itemize}
\item Pick a couple of pairs to show and explain what they did with the code. 
\item Give students exit tickets, containing example code similar to the code in these lessons, get them to explain what
  it does.
\item Give students code which is meant to do one thing, but which contains errors, get them to explain where the errors
  are.
\item Give students code which is meant to do one thing, but which contains errors, get them to fix the code and make it
  run. 
\item Get students to systematically sabotage each others' code and then ask the original author to fix (make sure they
  save the original code separately first).
\item Get them to complete a ``two stars and a wish'' exit ticket.
\item Check whole class understanding with quick fire questions about different parts of the code. 
\item Use Red/Amber/Green cards to answer multiple choice questions about the code.
\end{itemize}


\section*{Stretch \& Challenge}
\begin{itemize}
\item If they have completed the Gold Challenge then students could then be asked to supply random data to the various
  block types which are selected randomly from the list, it should be possible to use an if statement (or multiple ones)
  to choose the values between which the data runs, dependent upon the type of block.  
\end{itemize}

\section*{Support}
\begin{itemize}
\item Some children may need reassurance if they have lots of the errors in the code so pair children with appropriate
  partners.
\item Insist that students must have asked 3 of their compatriots for help before consulting with the teacher.  
\item Get students to make use of the Python Documentation which is accessible through the help menu (a web connection
  is often required for this). Students are to document the searches they have used prior to asking for help.
\item Teach students to use Google to search for errors when they need help.
\item Teach students to use StackOverflow to search for help.
\item Devise screencasts explaining vital concepts which are to be used in a lesson.
\item Get students to create screencasts, explaining the concepts which can then be accessed by their colleagues during
  subsequent lessons. 
\end{itemize}

\section*{Assessment Opportunities}
\begin{itemize}
\item Photographs taken of the code to assess understanding. 
\item Set student's a specific goal to achieve, and then assess their resulting code against the initial requirements
  you set.
\item Ask students to design code, to fulfil a specific task.
\item Get students to fully annotate their code, assess their work based on the quality of their explanations.
\item Give students a partially complete piece of code and ask them to complete it to fulfil a given task.
\item Children explain what they have done in an evaluation sheet or Computing log.  
\end{itemize}

%\section*{Key Vocabulary}

%\section*{Curriculum Links}

\section*{Resources}
\begin{itemize}
\item Snowdon map 
\item Minecraft Blocks Sheet
\item Snowdon Worksheet 2
\end{itemize}

%\section*{Useful Links}

\end{document}

%%% Local Variables:
%%% mode: latex
%%% TeX-master: t
%%% End:
