
\documentclass{geocraft-lesson-plan}

\usepackage[pdftex,
             pdfauthor={Sarah Zaman \& Dave Ames},
            pdftitle={GeoCraft Lesson Plan: Programming Snowdon Part
              2- Iteration},
            pdfcreator={LaTeX with hyperref and listings},
            urlbordercolor={1 1 1}]{hyperref}


\begin{document}

\mytitle{Programming Snowdon}
\subtitle{Part 2 - Iteration}
%\agerange{5/6/7}
\duration{1 hour}

\section*{Overview}
This is one of a series of lessons aimed at using Minecraft and Python
on the Raspberry Pi to develop students’ basic understanding of
programming. In this series we use the area around Mount Snowdon as
our location, and build up gradually to the point where students will
have enough knowledge to make changes to the environment on Snowdon.  

\section*{Objectives}
\begin{itemize}
\item Use sequencing in programs.
\item Designs simple algorithms using loops.
\item Uses loops within a program
\item Detect and correct errors in algorithms and programs.
\item Use logical reasoning to explain how a simple algorithm works. 
\item Design and debug programs that accomplish certain goals.
\end{itemize}

\section*{Video Starter}
\begin{itemize}
\item YouTube explanation of iteration (by Mark Zuckerberg):
 {\textcolor{greenish} 
  {\url{ https://m.youtube.com/watch?v=mgooqyWMTxk}}} 
\end{itemize}

\section*{Unplugged Starter}
\begin{itemize}
\item Use this resource from code.org, to help students grasp the
  concept of iteration (loops): 
  {\textcolor{greenish}
    {\url{https://code.org/curriculum/course2/5/Activity5-GettingLoopy.pdf}}} 
\item If you feel that the students need to spend longer on loops and 
  iteration, there is a complete lesson here: 
  {\textcolor{greenish}
    {\url{https://code.org/curriculum/course2/5/Teacher}}} 
\end{itemize}

\section*{Introduction to Minecraft on the Raspberry Pi}
\begin{itemize}
\item Talk through the introduction sheet with the children to help
  familiarise them with how to start programming in Minecraft Pi. 
\item Allow pupils time to tinker and orientate themselves in the
  Minecraft game. Build a colourful house. 
\end{itemize}

\section*{Main Activity}
\begin{itemize}
\item Explain to children they will be entering Python code to make a
  path which follows the player as they go up and down the mountain. 
\item Ask pupils to begin entering the code and then remind pupils how
  to run it. 
\item Children enter code and attempt the challenges to develop and
  extend their understanding of the code they’ve just written. 
\item Explain to the children they will be entering Python code to
  make a random trail. 
\item Discuss how random commands in Python work both how to choose a
  random number or choose a random item from a list. 
\item Pupils enter the code and attempt the challenges. 
\end{itemize}

\section*{Plenary}
\begin{itemize}
\item Pick a couple of pairs to show and explain what they did with
  the code. 
\item Check whole class understanding with quick fire questions about
  different parts of the code. 
\end{itemize}


\section*{Stretch \& Challenge}
\begin{itemize}
\item See the Gold Challenge
\end{itemize}

\section*{Support}
\begin{itemize}
\item Some children may need reassurance if they have lots of the
  errors in the code so pair children with appropriate partners. 
\end{itemize}

\section*{Assessment Opportunities}
\begin{itemize}
\item Photographs taken of the code to assess understanding. 
\item Children explain what they have done in an evaluation sheet or
  Computing log.  
\end{itemize}

%\section*{Key Vocabulary}

%\section*{Curriculum Links}

\section*{Resources}
\begin{itemize}
\item Snowdon map 
\item Introductory Sheet
\item Snowdon Worksheet 2
\end{itemize}

%\section*{Useful Links}

\end{document}

%%% Local Variables:
%%% mode: latex
%%% TeX-master: t
%%% End:
