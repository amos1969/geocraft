
\documentclass{geocraft-lesson-plan}

\usepackage[pdftex,
            pdfauthor={Sarah Zaman \& Dave Ames},
            pdftitle={GeoCraft Lesson Plan: Programming Snowdon Part 1 - Sequencing},
            pdfcreator={LaTeX with hyperref and listings},
            urlbordercolor={1 1 1}]{hyperref}


\begin{document}

\mytitle{Programming Snowdon}
\subtitle{Part 1 - Sequencing}
\duration{1 hour}

\section*{Overview} This is one of a series of lessons aimed at using Minecraft and Python on the Raspberry Pi to
develop students’ basic understanding of programming. In this series we use the area around Mount Snowdon as our
location, and build up gradually to the point where students will have enough knowledge to make changes to the
environment on Snowdon.  

\section*{Objectives}
\begin{itemize}
\item Use sequencing in programs.
\item Call built in functions to perform specific operations.
\item Detect and correct errors in algorithms and programs.
\item Use logical reasoning to explain how a simple algorithm works. 
\item Design and debug programs that accomplish certain goals.
\item Realise that calling the same function with different parameters will cause it to have different effects. 
\end{itemize}

\section*{Python constructs used}
\begin{itemize}
\item Use an external code library.
\item Create a variable and assign a value to it.
\item Call a function which is predefined in a library (access it using dot notation).
\item Pass parameters into a function when it is called.
\end{itemize}

\section*{MinecraftPi API Functions used}
\begin{itemize}
\item Import the main Minecraft module.
\item Import the blocks module.
\item Call the create function to establish a connection to Minecraft and store the link in a variable.
\item Use the player get position function to get x, y, and z coordinates of the player.
\item Call the setBlock function to set a specific block type at specific coordinates.
\item Pass data to a block to change its' behaviour.
\item Use the setBlocks function to create a larger cuboid.
\end{itemize}

\section*{Introduction to Snowdon Theme}
\begin{itemize}
\item Ask pupils to find Mount Snowdon on a map, or fly to it from your own location using Google Earth. 
\item  Where is it in relation to your school? Ensure they know that Snowdon is a mountain (1085m) and is part of the
  Snowdonia National Park.  
\end{itemize}

\section*{Video Starter}
\begin{itemize}
\item YouTube explanation of sequencing: 
  {\textcolor{greenish}
    {\url{https://www.youtube.com/watch?v=FHsuEh1kJ18}}}
\end{itemize}

\section*{Unplugged Starter}
You may need to register with Barefoot Computing to access the following resource, don't worry it's free.
\begin{itemize}
\item Have a look at some of the following activities:
  {\textcolor{greenish}
    {\url{http://barefootcas.org.uk/barefoot-primary-computing-resources/concepts/programming/sequence/}}}
\end{itemize}

\section*{Introduction to Minecraft on the Raspberry Pi}
\begin{itemize}
\item Allow pupils time to tinker and orientate themselves in the Minecraft game. Perhaps get them to build a colourful house. 
\item Once they've spent some time just playing with Minecraft, get them to open Idle3 from the Programming section of
  the Menu.
\item Explain that this starts the Python Shell which is where any messages from Python will be displayed. For instance
  if they introduce a bug into their code this is probably where the error message will display.
\item Explain that they need to create a new file to hold their Python code. They can do this by clicking on the File
  Menu in the Python Shell window and selecting ``New File''. This should open a new window. 
\item In this new window they need to click on the File Menu and select ``Save As''. Call the file by their name and
  click Okay.
\item When they start writing their code they need to do it in this new, empty file, which they have just created.
\end{itemize}

\section*{Related Activities}
\begin{itemize}
\item Minecraft uses 3D Cartesian Coordinates, you may need to discuss how these work with students.
\item The x-axis and z-axis run parallel to the floor, the y-axis runs vertically (positive y is up).
\item It may be useful to get students to physically model and plan out what they are doing in the Minecraft world. This
  can be done using things like, squared paper, multilink cubes and Lego. 
\item Work in Maths on volume might usefully be linked to this.
\end{itemize}


\section*{Main Activity}
\begin{itemize}
\item This activity focuses on sequencing, in other words putting Python instructions in the correct order, to cause a
  specific goal to be achieved.
\item The first part is specifically about the students becoming familiar with how to write a Python program which will
  talk to Minecraft on the Raspberry Pi and cause a block to appear.
\item The first program does the following.
\item Firstly we import the Minecraft module into our program, which contains the functions (technically methods) which
  allow us to talk to Minecraft.
\item The second line imports the blocks module, which gives us access to the various blocks which are available to us
 programmatically in MinecraftPi. 
\item We then have a line which creates a new variable called \textbf{mc} which holds the ``connection'' to the
  Minecraft game. The right hand side of the equals sign features a function which creates this.
\item The fourth line of code also creates a new variable called \textbf{pos}. The right hand side of the equals sign is
  a function which gets the coordinates of the player. These are then stored in the variable.
\item We then have a number of lines of code, which create x, y and z variables to hold the corresponding values based
  on the player's position.
\item The final line of code uses the \textbf{setBlock()} function, which in this case takes 4 parameters. The x, y and
  z coordinates at which to place the block, and the type of block to place there.
\item Running the code should make a Gold block appear next to the player's feet.
\item The \textbf{Bronze Challenge} gets the student to investigate what making changes to the coordinates has on where
  the block will appear. It might be useful to make changes of the form \textbf{x+5} rather than specific numbers, here,
  so that the blocks appear near to the player, rather than at random positions. (\textbf{HINT:} The player's current
  coordinates in the Minecraft world are displayed in the top left hand corner of the window.)
\item The \textbf{Silver Challenge} gets the student to experiment with different block types. There is a list of all of
  the available blocks in the General Files.
\item The \textbf{Gold Challenge} gets the user to place a TNT block, but also pass in some data to that block (did you
  notice the \textbf{, 1} after the block type is defined. The TNT block which is then created, will explode when you
  mine it.
\item The second example program contains many lines of code which are repeated from the first. The new variables x1, y1
  and z1 are the coordinates of a second point in the Minecraft world. The two points (x, y, z) and (x1, y1, z1) are the
  opposite vertices of a cuboid which is created in the last line of code by the \textbf{setBlocks()} function (notice
  the extra \textbf{s} at the end of the function name.
\item The \textbf{Bronze Challenge} asks the students to change the size of the cuboid that is created.
\item The \textbf{Silver Challenge} asks the students to create buildings by making a cuboid of air inside another, they
  may need to think carefully about how to position the air block relative to the solid one.
\item The \textbf{Gold Challenge} then asks them to use repeated code to create a row of holiday homes on Snowdon. (This
  is much easier to do with functions, but the students may not have covered these yet.)
\end{itemize}

\section*{Plenary - Suggested Activities}
\begin{itemize}
\item Pick a couple of pairs to show and explain what they did with
  the code. 
\item Check whole class understanding with quick fire questions about
  different parts of the code. 
\end{itemize}


\section*{Stretch \& Challenge}
\begin{itemize}
\item Can students create an entire village using code?
\end{itemize}

\section*{Support}
\begin{itemize}
\item 
\end{itemize}

\section*{Assessment Opportunities}
\begin{itemize}
\item Photographs taken of the code to assess understanding. 
\item Children explain what they have done in an evaluation sheet or Computing log.  
\end{itemize}

%\section*{Key Vocabulary}

%\section*{Curriculum Links}

\section*{Resources}
\begin{itemize}
\item Snowdon map and Introductory Sheet
\item Snowdon worksheet 1
\end{itemize}

%\section*{Useful Links}

\end{document}

%%% Local Variables:
%%% mode: latex
%%% TeX-master: t
%%% End:
