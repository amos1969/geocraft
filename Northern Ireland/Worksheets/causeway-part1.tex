
% causeway-part1.tex - Our first LaTeX example!

\documentclass{geocraft-worksheet}

\usepackage[pdftex,
             pdfauthor={Sarah Zaman \& Dave Ames},
            pdftitle={GeoCraft: Programming the Giant's Causeway Part 1 - Sequencing},
            pdfcreator={LaTeX with hyperref and listings}]{hyperref}

\begin{document}

\title{Programming the Giant's Causeway}
\subtitle{Part 1 - Sequencing}

\date{}

\maketitle

\pagenumbering{gobble}

\section{Place a Gold Block}
\lstset{language=Python}\vspace{-0.5cm}

\noindent% Line above MUST be blank
\tikzmark{Start}%
\lstinputlisting{01-gold.py}\vspace{0.2cm}
\tikzmark{End}
\AddBackgroundImage{Start}{End}{giants-causeway}%
%
\vspace{0.2cm}

\noindent This code creates a stone block at your feet.

\begin{itemize}
\item\textbf{Bronze Challenge:} Change the x, y, and z values so the block
appears next to you. \newline Try \textbf{x + 2} or \textbf{y + 3}. 

\item\textbf{Silver Challenge:} Try different blocks. See if you can
  place a \textbf{Gold Block} next to you. 

\item\textbf{Gold Challenge:} Use
\lstinline{ mc.player.setPos(50, 50,50) }
 to teleport Steve around the map. Change the numbers to try
  different places.
\end{itemize}

\section{Place a Larger Block}\vspace{-0.5cm}

\noindent% Line above MUST be blank
\tikzmark{Start}%
\lstinputlisting{02-diamond.py}\vspace{0.2cm}
\tikzmark{End}
\AddBackgroundImage{Start}{End}{giants-causeway}%
%
\vspace{0.5cm}

\noindent The \textbf{x1, y1, z1} are the coordinates of the other corner of the
cuboid. 

\begin{itemize}
\item\textbf{Bronze Challenge:} Can you change the size of the large block?

\item\textbf{Silver Challenge:} Can you create a number of cuboids of
  different sizes around you?

\item\textbf{Gold Challenge:} Can you create a block of air inside
  another block which you have created, to make a hollow block?
\end{itemize}




\end{document}

%%% Local Variables: 
%%% coding: utf-8
%%% mode: latex
%%% TeX-master: t
%%% End: 
