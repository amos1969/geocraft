
% causeway.tex - Our first LaTeX example!

\documentclass{geocraft-worksheet}

\usepackage[pdftex,
             pdfauthor={Sarah Zaman \& Dave Ames},
            pdftitle={GeoCraft: Programming the Giant's Causeway Part 2 - Iteration},
            pdfcreator={LaTeX with hyperref and listings}]{hyperref}

\begin{document}

\title{Programming the Giant's Causeway}
\subtitle{Part 2 - Iteration}

\date{}

\maketitle

\pagenumbering{gobble}

\setcounter {section} {2}


\section{Make Paths Which Follow you Around}\vspace{-0.5cm}
\lstset{language=Python}

\noindent% Line above MUST be blank
\tikzmark{Start}%
\lstinputlisting{03-stones.py}\vspace{0.2cm}
\tikzmark{End}
\AddBackgroundImage{Start}{End}{giants-causeway}%
%

\noindent This code creates a stone block path beneath your feet.

\begin{itemize}
\item\textbf{Bronze Challenge:} Change the block type to be exploding
  TNT: \lstinline{block.TNT.id, 1}

\item\textbf{Silver Challenge:} Change the coordinates to leave a
  trail in different places, try \textbf{above your head}. Try
  placing multiple blocks at a time.

\item\textbf{Gold Challenge:} By placing more than one block leave a
  pathway made of gold and diamonds.

\end{itemize}

\section{Making Random Trails}\vspace{-0.5cm}


\noindent% Line above MUST be blank
\tikzmark{Start}%
\lstinputlisting{04-random-trails.py}\vspace{0.1cm}
\tikzmark{End}
\AddBackgroundImage{Start}{End}{giants-causeway}%
%

\noindent The import random gives us access to Python's random number
functions, which we use to place random types of block. We can also
use \lstinline{random.choice} to choose a random item in a list.

\begin{itemize}
\item\textbf{Bronze Challenge:} Change the code to make it
  select a random \lstinline{STONE_SLAB} type (they are numbered 0 to 5).

\item\textbf{Silver Challenge:} Change the code to make it
  place a random \lstinline{WOOL} block colour (the colours run from 0 to 15).

\item\textbf{Gold Challenge:} Use \lstinline{random.choice([block.STONE.id, block.DIRT.id])} 
to place a random block. Add different block types to the list to make it more exciting.
\end{itemize}


\end{document}

%%% Local Variables: 
%%% coding: utf-8
%%% mode: latex
%%% TeX-master: t
%%% End: 
