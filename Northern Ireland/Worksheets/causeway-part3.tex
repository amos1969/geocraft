
% causeway-part3.tex - Our first LaTeX example!

\documentclass{geocraft-worksheet}

\usepackage[pdftex,
            pdfauthor={Sarah Zaman \& Dave Ames},
            pdftitle={GeoCraft: Programming the Giant's Causeway Part 3 - Selection},
            pdfcreator={LaTeX with hyperref and listings}]{hyperref}

\begin{document}

\title{Programming the Giant's Causeway}
\subtitle{Part 3 - Selection}

\date{}

\maketitle

\pagenumbering{gobble}


\setcounter {section} {4}

\section{Make a Bridge Over Water}\vspace{-0.5cm}
\lstset{language=Python}

\noindent% Line above MUST be blank
\tikzmark{Start}%
\lstinputlisting{06-bridge.py}\vspace{0.1cm}
\tikzmark{End}
\AddBackgroundImage{Start}{End}{giants-causeway}%
%
\vspace{-0.1cm}
\noindent This code creates a stone block path beneath your feet when
you are above water.\vspace{-0.1cm}

\begin{itemize}
\item\textbf{Bronze Challenge:} Change the block type to make an Iron Bridge.

\item\textbf{Silver Challenge:} Change the code so that you create a
  bridge which is 3 blocks wide (you can use Stone or Iron or some
  other type of block).

\item\textbf{Gold Challenge:} Change the code so that the bridge is 3 blocks wide in the North/South direction
and in the East/West direction, like a cross, when you walk across the water.


\end{itemize}

\section{Making Random Stacks}\vspace{-0.5cm}

\noindent% Line above MUST be blank
\tikzmark{Start}%
\lstinputlisting{07-stacks.py}\vspace{0.1cm}
\tikzmark{End}
\AddBackgroundImage{Start}{End}{giants-causeway}%
%
\vspace{-0.1cm}
\noindent If the block under your feet is grass, then this code will
place a 4 high stack of grass blocks next to you.\vspace{-0.1cm}

\begin{itemize}
\item\textbf{Bronze Challenge:} Can you change the code to make it
  place a random height stack of blocks?

\item\textbf{Silver Challenge:} Can you change the code to make it
  place 3 different random height stacks next to each other?

\item\textbf{Gold Challenge:} Now change it so that it places 9 random
  stacks in a square grid behind you as you move.
\end{itemize}


\end{document}

%%% Local Variables: 
%%% coding: utf-8
%%% mode: latex
%%% TeX-master: t
%%% End: 
