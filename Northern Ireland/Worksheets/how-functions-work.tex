
% how-functions-work.tex - Our first LaTeX example!
\documentclass{geocraft-worksheet-multipage}

\usepackage[pdftex,
             pdfauthor={Sarah Zaman \& Dave Ames},
            pdftitle={GeoCraft: Programming the Giant's Causeway - How
              Functions Work},
            pdfcreator={LaTeX with hyperref and listings}]{hyperref}

\begin{document}
\title{Programming the Giant's Causeway}
\subtitle{How Functions Work}

\date{}
\maketitle

\pagenumbering{gobble}

\lstset{language=Python}

\noindent% Line above MUST be blank
\tikzmark{Start}%
\lstinputlisting{09-functional.py}
\tikzmark{End}
\AddBackgroundImage{Start}{End}{giants-causeway}%
%
\vspace{0.2cm}

The code does lots of things, but mainly we have broken it down into
logical sections (called functions) which we can use anywhere in the
code, as building blocks to make more complicated programs. Our
example contains definitions for 3 functions called: \textbf{main,
  stacks} and \textbf{stacker}. \vspace{0.5cm}


The indented python after each use of the word \textbf{def} is called
a \textbf{function definition}. It is where we tell python what code to run
when we see each function name.  \vspace{0.5cm}


When we run the program, it goes through the normal process of
importing the libraries/modules we are using and then establishes the
connection to Minecraft as normal (\textbf{lines 1, 2, and 3}). Then
we come to \textbf{line 5} which contains the code 
\textbf{def main():} this is the line which tell Python: here is a
\textbf{def}inition for the function called \textbf{main}, the
brackets are also part of the definition (if we want to pass any
values to the function to use we do it in there) and the colon tells
Python that the next lines will be indented and are going to contain
the \textbf{body} of our function. \vspace{0.5cm}


The \textbf{main} function contains code for a while loop which runs
indefinitely, inside the loop we constantly get the position of the
player, create/update variables corresponding to the coordinates, and
pass those current coordinates to another function called
\textbf{stacks}. When we first run the program, Python will see the
function definition and the body of the function but will \textbf{NOT}
run any code at this point. Instead it notes down where to find the
definition for main, and what code to run when it is
called. \vspace{0.5cm}

\newpage

Effectively what you are doing with a function definition is
extending Python by creating your own commands, this one is called
\textbf{main} and to run it we would use \textbf{main()}. Which is the
code which is found on the very last line of the program. Python notes
down where all 3 of the \textbf{function definitions} are and then
reaches the final line of the program which calls the \textbf{main} function
and starts everything going. We have called it \textbf{main} because
it contains the main part of the program we want to run, it doesn't
actually matter what the function is called (we could just as easily
name it Harry, Ron, Hermione, Luke, Leia, or Han) as long as we
remeber when we get to the end of the code what we called it and use
that name to call it. Convention is to name it something relevant to
the program rather than a fictional name, though. 
\vspace{0.5cm} 


When \textbf{main()} is called, the code inside the definition runs
and at the end of each time through the loop (on each iteration) the
function called \textbf{stacks} is called, and is passed the three
coordinates which we have set. \textbf{Stacks} is another of our
functions which we have defined. The \textbf{function definition}
contains 4 lines of code, each of these lines is a call to another
of our functions \textbf{stacker}, which we call each time with
slightly different versions of the coordinates.  \vspace{0.5cm}


When the function \textbf{stacker} is called the first time, we give it
the coordinates of the player, with the x-coordinate and the
z-coordinate reduced by 3. The line of code inside \textbf{stacker}
runs and creates a 5 block high stack of blocks on these
coordinates. Then without any extra code being added it automatically
goes back to the point in the code it was called from (it
\textbf{returns}) and the next line of code is executed. This is again
a call to the \textbf{stacker} function with modified coordinates, it
returns and the next line executes, until all 4 lines within the
\textbf{stacks} function have run.  \vspace{0.5cm}


At this point we have reached the end of the \textbf{stacks} function,
so the code \textbf{returns} to where it was called from (the last
line of the while loop inside the main function. Since the last line
of the loop has been reached, the program then checks whether we can
run the loop again (because it says \textbf{True} it will always run)
and restarts the loop. The code in the loop then gets the player's
current position again, sets the variables for the coordinates and
calls the \textbf{stacks} function with our new values.  \vspace{0.5cm}


This process will repeat for ever until we tell Python that we want
the loop to exit (hold the control key down and press c in the Python
Shell) or we run a new Python program, which will restart the Shell.
\vspace{0.5cm} 


\end{document}

%%% Local Variables: 
%%% coding: utf-8
%%% mode: latex
%%% TeX-master: t
%%% End: 
