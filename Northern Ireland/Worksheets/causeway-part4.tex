
% causeway-part4.tex - Our first LaTeX example!

\documentclass{geocraft-worksheet-multipage}

\usepackage[pdftex,
            pdfauthor={Sarah Zaman \& Dave Ames},
            pdftitle={GeoCraft: Programming the Giant's Causeway Part 4 - Functions},
            pdfcreator={LaTeX with hyperref and listings}]{hyperref}

\begin{document}
\title{Programming the Giant's Causeway}
\subtitle{Part 4 - Functions}

\date{}
\maketitle

\pagenumbering{gobble}

\setcounter {section} {6}

\section{Using Functions to Simplify Your Code}\vspace{-0.3cm}
\lstset{language=Python}

\noindent% Line above MUST be blank
\tikzmark{Start}%
\lstinputlisting{09-functional.py}\vspace{0.1cm}
\tikzmark{End}
\AddBackgroundImage{Start}{End}{giants-causeway}%
%

There is a more detailed explanation in the 'How Functions Work' sheet
but essentially our code runs as follows: 

\begin{enumerate}

\item Import the modules we need, and create a connection to
  Minecraft.

\item Define a function called \textbf{main} all of the indented code
  on the following lines is part of this definition and only gets run
  when we call the function using \textbf{main()}.

\item Define a function called \textbf{stacks}.

\item Define a function called \textbf{stacker}.

\item Call the function named \textbf{main}. Which means that the code
  inside \textbf{main} will now run.

\item Start the loop inside \textbf{main}, get the player's position,
  store it in each of the three variables.

\item Call the function \textbf{stacks} giving it the current position
  of the player.

\item The program now runs the code inside \textbf{stacks} using the
  coordinates we called it with. This code in turn calls the function
  \textbf{stacker} four times, passing it slightly different
  coordinates on each call.

\item Each time the code in \textbf{stacker} is run, the program then
  \textbf{returns} to where it was called from, ie inside
  \textbf{stacks}. It then moves to the next line of code and runs
  that. 

\item When the program gets to the end of the code in \textbf{stacks}
  it will return to where \textbf{stacks} was called from, ie the last
  line of \textbf{main}.

\item Since the code inside \textbf{main} is part of a loop, the loop
  then runs again (the next \textbf{iteration} of the loop) and
  everything begins again at step 6 above.

\end{enumerate}

Now see if you can work through each of the following challenges in
order to recreate as accurate a representation of the Giant's Causeway
as possible using Minecraft (we will have to make square stacks rather
than hexagonal ones unfortunately).


\begin{itemize}
\item\textbf{Bronze Challenge:} Change one of the earlier programs you
  created to make it use functions.

\item\textbf{Silver Challenge:} Change the code above to place 9
  random height stone stacks near to you

\item\textbf{Gold Challenge:} Change the code so that it only places
  the stacks if they are on top of stone.

\item\textbf{Platinum Challenge:} Change the code so that when you are
  stood in the middle of the peninsula on the Giant's Causeway Map it
  will place random height stacks above any existing stone blocks

\item\textbf{Diamond Challenge:} Change the code so that it detects
  when the stack will be near to water and ensures that the maximum
  possible height is reduced (this should make the stacks look more
  realistic).

\item\textbf{Ultimate Challenge:} Make the stacks look more realistic
  by using blocks such as \lstinline{MOSS_STONE} and anything else you can think of.

\end{itemize}


\end{document}

%%% Local Variables: 
%%% coding: utf-8
%%% mode: latex
%%% TeX-master: t
%%% End: 
