
\documentclass{geocraft-lesson-plan}

\usepackage[pdftex,
             pdfauthor={Sarah Zaman \& Dave Ames},
            pdftitle={GeoCraft Lesson Plan: Programming the Giant's Causeway Part 4 - Functions},
            pdfcreator={LaTeX with hyperref and listings},
            urlbordercolor={1 1 1}]{hyperref}


\begin{document}

\mytitle{Programming the Giant's Causeway}
\subtitle{Part 4 - Functions}
\duration{1 hour}

\section*{Overview} This is one of a series of lessons aimed at using Minecraft and Python on the Raspberry Pi to
develop students' basic understanding of programming. In this series we use the area around the Giant's Causeway in
Northern Ireland as our location, and build up gradually to the point where students will have enough knowledge to
recreate a rough version of the Causeway. As Minecraft is limited to using square based blocks only, by definition the
stacks which the students will create will be cuboids rather than Hexagonal Prisms but a similar looking structure is
easily within the capabilities of most students.

\section*{Objectives}
\begin{itemize}
\item Use sequencing in programs.
\item Designs simple algorithms using loops, and selection i.e. if statements.
\item Uses arithmetic operators, if statements, and loops, within programs.
\item Designs, writes and debugs modular programs using procedures.
\item Knows that a procedure can be used to hide the detail with sub-solution (procedural abstraction).
\item Detect and correct errors in algorithms and programs.
\end{itemize}


%\section*{Links to the Progression Pathways}
%\begin{itemize}
%\item 
%\end{itemize}

\section*{Links to the National Curriculum}
\begin{itemize}
\item 
\end{itemize}

\section*{New python constructs used}
\begin{itemize}
\item Creating a function with no parameters
\item Creating a function with multiple parameters
\end{itemize}

\section*{New MinecraftPi API Functions used}
\begin{itemize}
\item No new constructs, but all of the previous ones need to be utilised.
\end{itemize}

\section*{Video Starter}
\begin{itemize}
\item YouTube explanation of funtions: 
  {\textcolor{greenish}
    {\url{https://m.youtube.com/watch?v=0eo0ESEX9DE}}}
\end{itemize}

\section*{Unplugged Starter}
\begin{itemize}
\item There is a nice whole lesson idea from code.org, which will introduce the concept of a function:  
  {\textcolor{greenish}
    {\url{https://studio.code.org/unplugged/unplug5.pdf}}}
\end{itemize}

\section*{Scratch Starter}
\begin{itemize}
\item Use a Scratch activity to demonstrate how functions work, see:
  {\textcolor{greenish} {\url{http://wiki.scratch.mit.edu/wiki/Custom_Blocks}}}. 
\end{itemize}

\section*{Programming Minecraft Reminder}
\begin{itemize}
\item Remind students how to move in the game if necessary.
\item Remind them to start the Python Shell, using IDLE3 in the Programming section of the Raspberry Pi Main Menu.
\item Remind them to create a new file to hold their program, and save it with an appropriate name.  
\item Remind them that when they make a change they will need to run their program to ensure it works.
\end{itemize}

\section*{Related Activities}
\begin{itemize}
\item Model each individual function's effects using Squared paper and multilink cubes.
\item Do a walk-through at the function level of what each component will do.
\item Get students to use video cameras to create a walk-through of their program using multilink.
\end{itemize}

\section*{Main Activity}
\begin{itemize}
\item The code at the top is explained in the section below it (a more detailed explanation is available on the ``How 
  Functions Work'' worksheet.
\item Being able to trace the flow of a computer program containing functions/procedures seems to be a key skill
  associated with learning how to program a computer.
\item The \textbf{Bronze Challenge} asks the students to rewrite a previous program to use at least one function.
\item Some example versions of these programs can be found with the other code samples.
\item The \textbf{Silver Challenge} gets them to change the code in the example program to place 9 stacks instead of the
  current 4. Plus the new stacks need to have a random height (\textbf{HINT:} The \verb!b+5! in the stacker function is
  what sets the height, replace the \verb!+5! with \verb!+a_random_number! that you set yourself)
\item The \textbf{Gold Challenge} asks them to only place blocks if they're on top of stone, this is similar to the
  Bridge over Water task they did previously. Both this and the \textbf{Silver Challenge} should be done using
  functions. 
\item The \textbf{Platinum Challenge} is just a development of the previous one as are the last two.
\item All of the challenges build on work completed in the previous lessons.
\item There will be code examples supplied.
\end{itemize}

\section*{Plenary}
\begin{itemize}
\item Provide students with a framework to self assess their work.
\item Maybe get them to peer assess each other according to the same framework.
\item Put scenarios on the board, ask them to explain, how they would code up each scenario, to each other. 
\item A variety of quizzes and other formative assessment tools could be used to check their understanding of the
  underlying concepts.
\item Exit tickets, two stars and a wish, and other common plenary activities could also be used.
\end{itemize}


\section*{Stretch \& Challenge}
\begin{itemize}
\item Students who complete the main task can add to their programs in a variety of ways to try and make the whole
  environment as realistic as possible. Are there trees on that part of the coast? How can they make ``realistic''
  looking trees in code?
\end{itemize}

\section*{Support}
\begin{itemize}
\item Make sure the students utilise the various self help techniques that have been discussed over the last few
  lessons. 
\item These techniques are what will help the students to become self reliant and independent learners in Computing. 
\item At no point should you just tell students where their mistake is, if you fix their code then they will expect you
  to always fix it when they have a problem.
\end{itemize}

\section*{Assessment Opportunities}
\begin{itemize}
\item Students create a full write up of what they've done, how they designed their program, how they implemented and
  tested it, including an evaluation of what they've done.
\item Use the students' peer and self evaluations as the basis of the assessments.
\end{itemize}

\section*{Resources}
\begin{itemize}
\item Causeway map
\item Introduction sheet 
\item Giant’s Causeway Worksheet 4
\item How Functions Work Sheet
\end{itemize}


\end{document}

%%% Local Variables:
%%% mode: latex
%%% TeX-master: t
%%% End:
