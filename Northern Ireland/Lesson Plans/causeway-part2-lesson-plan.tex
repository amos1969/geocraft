
\documentclass{geocraft-lesson-plan}

\usepackage[pdftex,
             pdfauthor={Sarah Zaman \& Dave Ames},
            pdftitle={GeoCraft Lesson Plan: Programming the Giant's Causeway Part 2 - Iteration},
            pdfcreator={LaTeX with hyperref and listings},
            urlbordercolor={1 1 1}]{hyperref}


\begin{document}

\mytitle{Programming the Giant's Causeway}
\subtitle{Part 2 - Iteration}
\duration{1 hour}

\section*{Overview} This is one of a series of lessons aimed at using Minecraft and Python on the Raspberry Pi to
develop students' basic understanding of programming. In this series we use the area around the Giant's Causeway in
Northern Ireland as our location, and build up gradually to the point where students will have enough knowledge to
recreate a rough version of the Causeway. As Minecraft is limited to using square based blocks only, by definition the
stacks which the students will create will be cuboids rather than Hexagonal Prisms but a similar looking structure is
easily within the capabilities of most students.

\section*{Objectives}
\begin{itemize}
\item Designs simple algorithms using loops.
\item Uses loops within a program
\item Detect and correct errors in algorithms and programs.
\end{itemize}

\section*{Links to the Progression Pathways?}

\begin{itemize}
\item Understands that iteration is the repetition of a process such as a loop. (AL)
\item Can identify similarities and differences in situations and can use these to solve problems (pattern
  recognition). (GE) 
\item Uses and manipulates one dimensional data structures. (AB)
\item Detects and corrects syntactical errors. (AL)
\end{itemize}

\section*{New python constructs used}

\begin{itemize}
\item Import the random library
\item Use the random.randint(...) function
\item Use the random.choice() function
\item Use a while loop
\item Pass extra data into a function
\end{itemize}

\section*{New MinecraftPi API Functions used}

\begin{itemize}
\item Pass data for a block into the \textbf{setBlock(...)} function.
\end{itemize}

\section*{Video Starter}
\begin{itemize}
\item YouTube explanation of iteration (by Mark Zuckerberg): 
  {\textcolor{greenish}
    {\url{https://m.youtube.com/watch?v=mgooqyWMTxk}}}
\end{itemize}


\section*{Unplugged Starter}
\begin{itemize}
\item Use this resource from code.org, to help students grasp the concept of iteration (loops): 
  {\textcolor{greenish}
    {\url{https://code.org/curriculum/course2/5/Activity5-GettingLoopy.pdf}}}
\item If you feel that the students need to spend longer on loops and iteration, there is a complete lesson here:  
  {\textcolor{greenish}
    {\url{https://code.org/curriculum/course2/5/Teacher}}}
\end{itemize}

\section*{Programming Minecraft Reminder}
\begin{itemize}
\item Remind students how to move in the game if necessary.
\item Remind them to start the Python Shell, using IDLE3 in the Programming section of the Raspberry Pi Main Menu.
\item Remind them to create a new file to hold their program, and save it with an appropriate name.  
\item Remind them that when they make a change they will need to run their program to ensure it works.
\end{itemize}

\section*{Related Activities}
\begin{itemize}
\item Use squared paper to plan out structures/patterns which they might construct.
\item Give students cards with various lines of Python on, get them to sort them into the correct order to achieve
  specific activities.
\item Flow chart their ideas for various potential programs they could make using Python and Minecraft
\item Use Pseudocode to plan out their programs
\end{itemize}

\section*{Main Activity}
\begin{itemize}
\item Explain to children they will be entering Python code to make a path which follows the player around. 
\item The first piece of code is virtually identical to the code in the previous lesson. 
\item Essentially the only real changes are that the block appears at coordinate y-1 (which means under the player's
feet and that there is a line which reads \textbf{while True:}, this is the command which starts a loop running. The
\textbf{True} part means make this loop run forever. The indentation on the following lines is what tells Python that
these lines are part of the loop, they are essentially the lines of code which will be run over and over again. In
Python if we want to have code that isn't part of the loop, but follows after it, then we would unindent the code.
\item Make sure that the students run their code once it has been entered, and return to the Minecraft world to see what
effect it has.
\item The first set of challenges involve either changing the type of block which is left behind or the positioning of
  the blocks relative to the player. Putting two \textbf{setBlock()} commands inside the loop with slightly different
  coordinates will cause two blocks to appear each time (if they both have the same coordinates then the second block will
  replace the first block before you can see it appear).
\item There is also a new part of a command. In the \textbf{setBlock()} function after block.xxxxx.id, we can pass data
to the block to change either it's appearance or behaviour by adding a comma and a number, details of the blocks which
can accept data and the effects of that data are included on the second page of the Minecraft Blocks Sheet. So in the
\textbf{Bronze Challenge}, the whole line of code becomes \textbf{mc.setBlock(x, y-1, z, block.TNT.id, 1)}, to detonate the
resulting TNT block we need to mine it.
\item Moving on to the second half of the lesson, we add in the idea of random functions, to produce randomised paths
  under the player's feet. 
\item The \textbf{import random} gives us access in this program to the built in Python Random Number Library, which
  contains functions we can use to generate a random number between two other values, or pick a random item from a list
  or a whole host of other things. In this section we do those two things, set a variable to be random value each time
  through the loop using \textbf{random.randint()} and then pass this value as \textbf{data} to the block we are
  setting. We also ask them to choose a random item from a list using the \textbf{random.choice()} function, which
  returns one value from the list which is passed into it.
\item The main part of the program and both the \textbf{Bronze} and \textbf{Silver Challenges} ask the user to pass a
  random integer as data to various block types. The \textbf{Silver Challenge} in particular produces quite striking
  results. 
\item The \textbf{Gold Challenge} takes a different approach and asks the student to create a list of different blocks,
which they then have to randomly select one of, each time through the loop. Although there are a number of ways this
could be done in Python we have directed students to use the \textbf{random.choice()} function here as it is the most
straightforward for this task. The students can be directed to add extra block types to the list they create (the bit
between the square brackets). They could also create a variable to hold the list, and then just pass the variable name
in. Create a list variable like this: \begin{verbatim}list_of_blocks = [block.xx.id, block.yy.id, block.zz.id]\end{verbatim}  
and then get to it in the \textbf{setBlock()} function like this  
\begin{verbatim}mc.setBlock(x, y-1, z, random.choice(list_of_blocks))\end{verbatim}  
\end{itemize}

\section*{Plenary - Suggested Activities}
\begin{itemize}
\item Get students to write a program which will place 4 blocks at random distances from the player, by using
  \textbf{random.randint()} to create values which are added or subtracted from x, y or z, in the \textbf{setBlock()}
  function. 
\item Give students code with bugs in them and get them to identify the bugs.
\item Create a Kahoot quiz using fragments of code (related to today's exercises) that have syntax errors in them.
\item Give out a number of cards one set between two, get the students to create programs by assembling the cards in the
  correct order, to achieve specific goals.
\item Code dominoes: Create sets of code dominoes which require the students to match pieces of code to the result of
  running that code (either outputs or explanations).
\item Exit tickets, students annotate what a fragment of code on their exit ticket (pre-printed) does.
\item Exit tickets, students annotate what a fragment of code outputs..
\item Two stars and a wish.
\item Peer review each others work and give it feedback, before doing the same task on their own code.
\end{itemize}

\section*{Stretch \& Challenge}
\begin{itemize}
\item If the \textbf{Gold Challenge} is too easy, get them to make a randomly coloured ``Disco Floor'' that appears
  around the players feet (10x10 or 15x15) works well.
\item Perhaps get the floor to appear randomly wherever the player walks.
\end{itemize}

\section*{Support}
\begin{itemize}
\item When students run into difficulties, such as bugs and syntax errors. Try to teach them to be resilient and self
  directed. 
\item If they have an error message get them to paste it into Google to look for similar errors
\item Teach them to use the Python Documentation website.
\item If they experience logic errors (the code runs but in unexpected ways) get them to try some variation on Rubber
  Duck Debugging: {\textcolor{greenish} {\url{https://en.wikipedia.org/wiki/Rubber_duck_debugging}}} Perhaps talking
  each other through the code line by line.
\item You as a teacher should \textbf{NOT} under any circumstances seek to fix their code. See Phil Bagge on Combating
  Learnt Helplessness: 
    {\textcolor{greenish} {\url{http://philbagge.blogspot.co.uk/2015/02/eight-steps-to-promote-problem-solving.html}}} 
\end{itemize}

\section*{Assessment Opportunities}
\begin{itemize}
\item Get students to create a screencast of what they've done, edit it and include a narrative created by them.
\item Or do a similar thing with a video camera.
\item Ask children to create a worksheet explaining how to achieve the task they've achieved to someone who hasn't done
  so before.
\item Get students to evaluate each others work according to some criteria you give them, then evaluate their own
  according to the same criteria
\item Give them a fix the bugs quiz, with similar code to that which they have been writing, but with bugs in, ask them
  to find and potentially fix the bugs.
\end{itemize}

\section*{Resources}
\begin{itemize}
\item Causeway map
\item Introduction sheet (Getting Started for Students)
\item Giant’s Causeway Worksheet 2
\end{itemize}

\end{document}

%%% Local Variables:
%%% mode: latex
%%% TeX-master: t
%%% End:
