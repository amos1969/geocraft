
\documentclass{geocraft-lesson-plan}

\usepackage[pdftex,
             pdfauthor={Sarah Zaman \& Dave Ames},
            pdftitle={GeoCraft Lesson Plan: Programming the Giant's Causeway Part 3 - Selection},
            pdfcreator={LaTeX with hyperref and listings},
            urlbordercolor={1 1 1}]{hyperref}


\begin{document}

\mytitle{Programming the Giant's Causeway}
\subtitle{Part 3 - Selection}
\duration{1 hour}

\section*{Overview} This is one of a series of lessons aimed at using Minecraft and Python on the Raspberry Pi to
develop students' basic understanding of programming. In this series we use the area around the Giant's Causeway in
Northern Ireland as our location, and build up gradually to the point where students will have enough knowledge to
recreate a rough version of the Causeway. As Minecraft is limited to using square based blocks only, by definition the
stacks which the students will create will be cuboids rather than Hexagonal Prisms but a similar looking structure is
easily within the capabilities of most students.

\section*{Objectives}
\begin{itemize}
\item Designs simple algorithms using while loops, for loops, and selection.
\item Uses arithmetic operators, if statements, and loops, within programs.
\item Nest one construct inside another (if statement inside a while loop).
\item Detect and correct errors in algorithms and programs.
\end{itemize}

%\section*{Links to the Progression Pathways}
%\begin{itemize}
%\item Understands that iteration is the repetition of a process such as a loop. (AL)
%\item Uses a range of operators and expressions e.g. Boolean, and applies them in the context of program control. (AL) 
%\item Understands the difference between ‘While’ loop and ‘For’ loop, which uses a loop counter. (AL) (AB)
%\end{itemize}

\section*{New python constructs used}
\begin{itemize}
\item if statements
\item for loops
\end{itemize}

\section*{New MinecraftPi API Functions used}
\begin{itemize}
\item mc.getBlock(...) which returns the type of block at a give coordinate.
\end{itemize}

\section*{Video Starter}
\begin{itemize}
\item YouTube explanation of selection (by Bill Gates): 
  {\textcolor{greenish}
    {\url{https://m.youtube.com/watch?v=m2Ux2PnJe6E}}}
\end{itemize}

\section*{Unplugged Starter}
\begin{itemize}
\item Unplugged selection activity in pairs
  {\textcolor{greenish}
    {\url{http://code-it.co.uk/wp-content/uploads/2015/05/ConditionalSelection.pdf}}}
\end{itemize}

\section*{Scratch Starter}
\begin{itemize}
\item Use a Scratch activity to demonstrate how if statements
  work. See Barefoot materials. ({\textcolor{greenish}
    {\url{http://barefootcas.org.uk/}}})  
\end{itemize}

\section*{Programming Minecraft Reminder}
\begin{itemize}
\item Remind students how to move in the game if necessary.
\item Remind them to start the Python Shell, using IDLE3 in the Programming section of the Raspberry Pi Main Menu.
\item Remind them to create a new file to hold their program, and save it with an appropriate name.  
\item Remind them that when they make a change they will need to run their program to ensure it works.
\end{itemize}

\section*{Related Activities}
\begin{itemize}
\item Use 1cm squared paper and Multilink cubes to work through line by line what each program does, maybe keep track of
  the state of each variable at each point in the code, especially during loops.
\item Use 1cm squared paper to work out where, relative to the player, the random stacks will appear, or the bridge
  blocks.
\item Using suitable materials, get students to act out the program for real (outside or in a hall, perhaps).

\end{itemize}

\section*{Main Activity}
\begin{itemize}
\item Explain to children they will be entering Python code to make a bridge beneath their feet when they walk on water. 
\item Explain how an if statement works in Python and relate it back to Scratch, the indented part of the code in Python
  is the equivalent of the blocks found between the jaws of the if block in Scratch. 
\item As in the previous activity, the first part of the code is identical to the start of the code in the previous
  sections. The coordinates of the player are stored in the variables x, y and z.
\item The line which creates a new variable called \textbf{blockType} calls the function \textbf{mc.getBlock(x, y-1, z)}
  which returns the type of the block beneath the player's feet (the y-1 part of the coordinates).  
\item We then use an if statement to compare the value in the variable \textbf{blockType} with the value
  \textbf{block.WATER\_STATIONARY.id}, if they are the same then the code which is indented runs. That is, a Stone block
  is created under the player's feet. This only happens if the player is above a Stationary Water block.
\item \textbf{NOTE} the comparison uses double equals \textbf{==}. 
\item In other words if the player is walking on water then a stone block appears, otherwise it doesn't.
\item Note that this doesn't happen if the player is standing over a \textbf{WATER/WATER\_FLOWING} block. The student
  will need to add more code, to make this happen. There are a number of ways to do this, one way is to change the
  comparison part of the if statement to: 
\begin{verbatim}if blockType == block.WATER_STATIONARY.id or blockType == block.WATER.id:\end{verbatim} 
\item The challenges then ask the students to change the block types of the bridge and the shape of the pattern of
  blocks which are left under the player's feet (multiple calls to mc.setBlock(..) are needed for this).
\item The second section brings in yet another new construct, the \textbf{for} loop. A \textbf{while} loop runs an
  unspecified number of times, usually until something becomes true or something becomes false. A \textbf{for} loop in
  the other hand runs up to a specific number of times either according to a value you give to the loop, or by accessing
  each of the items in an object such as a loop, one at a time.
\item The code is nearly identical to the previous code, but this time we are checking for a grass block beneath the
  player's feet. If this is found then the indented code is run.
\item Firstly we set a new variable \textbf{max\_height} to be  3, we then call a \textbf{for} loop. With the code
  \begin{verbatim}for height in range(max_height):\end{verbatim} which does the following. Start by setting the \textbf{height} variable to
  0, then set a block at the coordinates given (including adding the height to y). Next set the \textbf{height} variable
  to be 1 and set a block at the new coordinate (1 up from the previous one). Set the height to be 2 and do the same,
  then because the \textbf{max\_height} variable had a value of 3 stop. (The in range part stops on a value one less than
  the value which is given to it)
\item In other words a 3 high stack is created next to wherever we walk.
\item To change the code for the challenges so that we can have random heights, we need to do two things, firstly add a
  line that reads \begin{verbatim}import random\end{verbatim}so that we can access the \textbf{random} number library
  again. We also need to change the line which currently reads \begin{verbatim}max_height = 3\end{verbatim} to read
  something like \begin{verbatim}max_height = random.randint(0, 10)\end{verbatim}The 0 and 10 can be any numbers you
  choose, the \textbf{randint()} function will pick random values between (and including) the two. 
\item The other challenges should be relatively straightforward to solve by this point (use multiple for loops one for
  each stack you are creating). 
\end{itemize}

\section*{Plenary - Suggested Activities}
\begin{itemize}
\item Add 10 bugs to a simple program, students have to identify the bugs as quickly as possible, and state what each
  one does.
\item Describe a piece of code that you want them to create to achieve a given goal, they create it.
\item Give them a piece of code which has a specific purpose (given to them) but which contains one or more logic
  errors, they have to find and fix the logic.
\item Give the students an algorithm described in a flow chart they have to implement it.
\item Give the students an algorithm described by pseudocode they have to implement it.
\item Give the students a working program, get them to describe what it does in natural english/with a flow chart/in
  pseudocode.
\end{itemize}


\section*{Stretch \& Challenge}
\begin{itemize}
\item What other aspects of if statements, for loops and random functions could they devise a project around?
\end{itemize}

\section*{Support}
\begin{itemize}
\item Referring back to ho the if statement and other constructs work in Scratch to relate them to the equivalent
  constructs in Python.
\item Explain their code line by line to another student, explaining what each part does and why it is there.
\item What happens to the rest of the program if they comment out a randomly selected line of code? Why?
\end{itemize}

\section*{Assessment Opportunities}
\begin{itemize}
\item Get them to peer assess two other students work, then assess their own using the same criteria.
\item Get the students to create an audio feedback file, where they have to describe what they have done with the code
  and what it does, without ever being able to show their code or it's outcome.
\item Get them to devise a series of criteria by which the could assess their own code, then apply them.
\item Write a report detailing what they have done, how, and what difficulties they had.
\item Create a screencast assessing someone else's work.
\end{itemize}

\section*{Resources}
\begin{itemize}
\item Causeway map
\item Introduction sheet 
\item Giant’s Causeway Worksheet 3
\end{itemize}


\end{document}

%%% Local Variables:
%%% mode: latex
%%% TeX-master: t
%%% End:
