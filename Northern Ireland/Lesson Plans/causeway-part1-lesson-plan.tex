
\documentclass{geocraft-lesson-plan}

\usepackage[pdftex,
             pdfauthor={Sarah Zaman \& Dave Ames},
            pdftitle={GeoCraft Lesson Plan: Programming the Giant's Causeway Part 1 - Sequencing}, 
            pdfcreator={LaTeX with hyperref and listings},
            urlbordercolor={1 1 1}]{hyperref}


\begin{document}

\mytitle{Programming the Giant's Causeway}
\subtitle{Part 1 - Sequencing}
\duration{1 hour}

\section*{Overview} This is one of a series of lessons aimed at using Minecraft and Python on the Raspberry Pi to
develop students' basic understanding of programming. In this series we use the area around the Giant's Causeway in 
Northern Ireland as our location, and build up gradually to the point where students will have enough knowledge to
recreate a rough version of the Causeway. As Minecraft is limited to using square based blocks only, by definition the
stacks which the students will create will be cuboids rather than Hexagonal Prisms but a similar looking structure is
easily within the capabilities of most students.

\section*{Objectives}

\begin{itemize}
\item Use sequencing in programs.
\item Detect and correct errors in algorithms and programs.
\item Use logical reasoning to explain how a simple algorithm works.
\item Design and debug programs that accomplish certain goals.
\end{itemize}

%\section*{Links to the Progression Pathways}

%\begin{itemize}
%\item Has practical experience of a high-level textual language, including using standard libraries when
%  programming. (AB) (AL) 
%\item Declares and assigns variables. (AB)
%\item Knows and uses functions appropriately. (AL) (AB)
%\item Detects and corrects syntactical errors. (AL)
%\end{itemize}

\section*{Python constructs used}

\begin{itemize}
\item Importing from a module. (Using an external code library)
\item Creating a variable and assigning a value to it. (Variable creation and assignment) 
\item Calling a function with no parameters. (Using a reusable piece of code - a function)
\item Calling a function with one or more parameters. (Passing parameters into a function)
\end{itemize}

\section*{MinecraftPi API Functions used}

\begin{itemize}
\item Importing the main Minecraft module.
\item Importing the blocks module.
\item Calling the create() function to establish a connection from the program to Minecraft.
\item Using the getPos() function to return the player's coordinates.
\item Accessing the x, y and z attributes of the object returned by the getPos() function.
\item Using the setBlock(...) function.
\item Creating different block types.
\item Using the setBlocks(...) function.
\item Dealing with the coordinate system in Minecraft.
\end{itemize}

\section*{Introduction to Northern Ireland Theme}
\begin{itemize}
\item Ask pupils to find Northern Ireland on on a map, or fly to it from your own location using Google Earth. Where is
it in relation to your school?  
\item Ensure they know that the Giants Causeway is an area of about 40,000 interlocking basalt columns, the result of an
ancient volcanic eruption. 
\end{itemize}

\section*{Video Starter}
\begin{itemize}
\item YouTube explanation of sequencing: 
  {\textcolor{greenish}
    {\url{https://www.youtube.com/watch?v=FHsuEh1kJ18}}}
\end{itemize}

\section*{Unplugged Starter} You may need to register with Barefoot Computing to access the following resource, don't
worry it's free.
\begin{itemize}
\item Unplugged sequencing activity in pairs  
  {\textcolor{greenish}
    {\url{http://barefootcas.org.uk/programme-of-study/understand-algorithms/ks1-crazy-character-algorithms-activity/}}} 
\end{itemize}

\section*{Introduction to Minecraft on the Raspberry Pi}
\begin{itemize}
\item Allow pupils time to tinker and orientate themselves in the Minecraft game. Perhaps get them to build a colourful house. 
\item Once they've spent some time just playing with Minecraft, get them to open Idle3 from the Programming section of
  the Menu.
\item Explain that this starts the Python Shell which is where any messages from Python will be displayed. For instance
  if they introduce a bug into their code this is probably where the error message will display.
\item Explain that they need to create a new file to hold their Python code. They can do this by clicking on the File
  Menu in the Python Shell window and selecting ``New File''. This should open a new window. 
\item In this new window they need to click on the File Menu and select ``Save As''. Call the file by their name and
  click Okay.
\item When they start writing their code they need to do it in this new, empty file, which they have just created.
\end{itemize}

\section*{Related Activities}
\begin{itemize}
\item Get students to work through activities which require them to create simple algorithms by putting instructions
  into the correct order (i.e. sequencing).
\item Barefoot CAS Viking Raid Animation Activity
  {\textcolor{greenish}
    {\url{http://barefootcas.org.uk/programme-of-study/use-sequence-in-programs/upper-ks2-viking-raid-animation-activity/}}}
\item Get students to work with and solve problems involving 3D coordinates.
\item Use Multilink Cubes (ask your Maths Department if they have any) to solve 3D problems made from cubes.
\end{itemize}

\section*{Main Activity}
\begin{itemize}
\item This first lesson is all about sequencing, i.e. putting instructions in the correct order to make things happen in
  the code. When the program is run it will step through the lines of code in order, each line will either cause some
  change in the current state of the program, some might create variables, some might send a message to Minecraft etc.
\item Python will keep track of where it is in the program and deal with the code accordingly. Later on we will create
  more complicated programs which won't just step line by line through the code which is written (but this is for a
  later lesson).
\item The initial program on the worksheet will create a Stone block next to the player's feet. It does this by
  detecting the player's coordinates and storing them into the variables x, y and z, before then creating a Stone block
  at those coordinates. You may need to explain 3D coordinates to the students (in Minecraft the x-axis and z-axis run
  parallel to the ground, whilst the y-axis is vertical).
\item There then follow a number of challenges, firstly to make the block appear in different positions relative to the
  player. By adjusting the x, y, and z variables inside the setBlock command.
\item The students are then asked to change the type of block which will appear, there is a sheet with a list of all
  possible valid block types in the General Files section of the download. Whichever block they choose to use, they need
  to use the name (all in capitals) in their code. It needs to replace the name STONE (which is in capitals) in the
  setBlock command, they still need to include the block. before it and the .id after it, so if they chose to use the
  GRASS block instead, that part of the command would read \textbf{block.GRASS.id} 
\item The Gold challenge introduces a new command which will teleport the player (aka Steve) around the map.
\item The second half of the lesson focuses on creating large blocks using the \textbf{setBlocks} command (note the
  extra \textbf{s} on the end of the command. You need to flag it up to the students as they will often miss it.
\item The x, y, z and x1, y1, z1 coordinates, constitute two diagonally opposite corners (vertices) of a cuboid, the
  bigger the difference between them, the bigger the cuboid. Which is the key to the first of the challenges.
\item The second challenge will require them to make repeated calls to the \textbf{setBlocks} command, but with
  different coordinates for each.
\item The final challenge will require them to think carefully about the coordinates of the solid block and the
  coordinates of the air block inside relative to the solid block.
\end{itemize}

\section*{Plenary - Suggested Activities}
\begin{itemize}
\item Pick a couple of pairs to show and explain what they did with the code. 
\item Check whole class understanding with quick fire questions about different parts of the code. 
\item Share good ideas with the rest of the class.
\item Try a two stars and a wish activity, students write down two things they've learned/enjoyed and one thing they
  wish they could learn more about/improve in their code.
\item Create a Kahoot or similar quiz to test their knowledge of the various things they've learnt.
\item Give them some find the bug/fix the code style activities based on the code they've worked with.
\item Something else? Please share any good ideas, via the GeoCraft Website.
\end{itemize}


\section*{Stretch \& Challenge}
\begin{itemize}
\item For those students finding the tasks too easy, set them the challenge of creating a building such as a castle,
  just using Python.
\item Alternately get them to write code which builds letters out of blocks so that they can spell their name.
\end{itemize}

\section*{Support}
\begin{itemize}
\item Some children may need reassurance if they have lots of the errors in the code.
\item Teach them to become independent learners using techniques such as 3 Before Me.
\item If they have an error message they don't understand, can they Google for an answer (try pasting the final line or
  lines of the error message into Google)?  
\item Are there solutions to similar problems on StackOverflow?
\item Can they find solutions to their problems using the Python Documentation on the Python website?
\end{itemize}

\section*{Assessment Opportunities}
\begin{itemize}
\item Photographs taken of the code to assess understanding. 
\item Children explain what they have done in an evaluation sheet or Computing log.   
\item Students make a screencast of their Minecraft creation, narrating what the code does.
\item What parts of the code did they complete? Keep a record of all possible parts and their progress.
\end{itemize}

\section*{Resources}
\begin{itemize}
\item Causeway map
\item Introduction sheet (Getting Started Guide for Students)
\item Giant’s Causeway Worksheet 1
\end{itemize}

\end{document}

%%% Local Variables:
%%% mode: latex
%%% TeX-master: t
%%% End:
