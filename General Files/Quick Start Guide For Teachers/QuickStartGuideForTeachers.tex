
% QuickStartGuideForTeachers.tex - Our first LaTeX example!

\documentclass{geocraft-worksheet-multipage}



\usepackage[pdftex,
             pdfauthor={Sarah Zaman \& Dave Ames},
            pdftitle={GeoCraft: Quick Start Guide for Teachers},
            pdfcreator={LaTeX with hyperref and listings},
            urlbordercolor={1 1 1}]{hyperref}


\begin{document}
\title{Quick Start Guide}
\subtitle{for Teachers}
\date{}

\maketitle

This guide assumes that you have an up to date image of the Operating
System for the Raspberry Pi, which will mean you are able to use
Python 3 to talk to Minecraft and all of the files will be located in
the correct places. If you run in to problems, please refer to the
Getting Started Guide for Teachers which contains much more detailed
instructions about how to get started. 

\section*{How to Download the Files}
\begin{enumerate}
\item From the GeoCraft Website, choose the resource pack that you wish to
use.
\item Download the zip file.
\item Copy it on to a pen drive.
\item Put the pen drive into the Raspberry Pi.
\item Click on Okay when asked about opening in the file manager.
\item Copy the zip file onto the Raspberry Pi's Desktop.
\item Right click and unzip the file.
\item Follow the steps in the next section.
\end{enumerate}

\section*{On the first go}
\begin{enumerate}
\item Go into the folder you unzipped previously by double clicking on
  it.
\item In the folder, find the Maps folder, and go into that folder.
\item Find the file called ResetGeoCraftMaps.py.
\item Right click on the file.
\item Select Properties.
\item Click on the Tab marked Permissions.
\item Click on the options to the right of the word Execute.
\item Select the word Anyone from the list.
\item Click Okay.
\item Move on to the next section now.
\end{enumerate}

\section*{After the first go}
\begin{enumerate}
\item Make sure that Minecraft is not running a map at the moment
  (Quit to title if it is).
\item Double click on the file called ResetEdinburghMaps.py.
\item In the window which pops up select Execute not Execute in terminal.
\item When the new window opens, read the warning and click the button
  if you wish to proceed.
\item This will place a clean version of the map you've downloaded
  into the right place for Minecraft to find it. 
\item It will replace any existing copies of the map with new ones,
  and thus erase any changes you've made to the map in Minecraft.
\end{enumerate}
\end{document}

%%% Local Variables: 
%%% coding: utf-8
%%% mode: latex
%%% TeX-master: t
%%% End: 
