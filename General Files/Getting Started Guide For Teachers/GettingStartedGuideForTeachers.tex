
% GettingStartedGuideForTeachers.tex - Our first LaTeX example!

\documentclass{geocraft-worksheet-multipage}



\usepackage[pdftex,
             pdfauthor={Sarah Zaman \& Dave Ames},
            pdftitle={GeoCraft: Getting Started Guide for Teachers},
            pdfcreator={LaTeX with hyperref and listings},
            urlbordercolor={1 1 1}]{hyperref}


\begin{document}
\title{Getting Started Guide}
\subtitle{for Teachers}
\date{}

\maketitle
 
\section*{How to download the files}
\begin{enumerate}
\item From the GeoCraft Website, choose the resource pack that you wish to
use.
\item Download the zip file.
\item Copy it on to a pen drive.
\item Put the pen drive into the Raspberry Pi. (See the section: Pen
  drives and the Model B Raspberry Pi)
\item Click on Okay when asked about opening in the file manager.
\item Copy the zip file onto the Raspberry Pi's Desktop.
\item Right click and unzip the file.
\item Either follow the steps on Updating the Raspberry Pi or look at
  the section on Getting the maps in the right place.
\end{enumerate}

\section*{Pen drives and the Model B Raspberry Pi}
If you have a newer model Raspberry Pi with 4 USB Ports just proceed
as normal. If you are using an older Model B Raspberry Pi then you
will only have two USB ports. If you have an empty USB port then you
can just plug the pen drive in and retrieve the files normally. If
like most  people you have a keyboard plugged in to one port and a
mouse plugged in to the other then you will need to unplug the
keyboard, plug the pen drive in retrieve the files, safely remove the
pen drive, and plug the keyboard back in.

\textbf{Potential Problems:}
\begin{itemize}
\item You may find that when plugging or unplugging something the
  screen freezes, this is usually due to a minor power surge crashing
  the Raspberry Pi. If this happens, pull out the power lead, reboot
  the Raspberry Pi and continue as normal (if it happened when you
  plugged in the pen drive, before you were able to transfer the
  files, then reboot with the pen drive connected, then transfer the
  files). 
\item You can't find a way to safely remove the pen drive. In this
  case you can either risk just pulling the pen drive out, or you can
  shut the Raspberry Pi down, remove the pen drive, reconnect the
  keyboard and reboot.
\end{itemize}


\section*{Update the Raspberry Pi (Optional)}
If you have lots of Raspberry Pis to update you are better off getting
one image sorted and copying that image onto the other SD Cards,
rather than doing this multiple times. Or you may find it easier to
download and install a brand new image to the SD Cards (See the
section: Putting a new image onto the SD Cards)
\begin{enumerate}
\item Make sure you are connected to the internet.
\item Click on the Menu.
\item Click on Accessories.
\item Click on Terminal to run it.
\item Or on older versions double click the LXTerminal Icon on the desktop.
\item Type \textbf{sudo apt-get update} and hit enter.
\item Wait for the script to finish running.
\item Type \textbf{sudo apt-get upgrade} and hit enter [Hit Enter
  again to confirm if you are asked to continue].
\item Wait for the script to finish running.
\item Type \textbf{sudo apt-get dist-upgrade} and hit enter [Hit Enter
  again to confirm if you are asked to continue]. 
\item Wait for the script to finish running.
\item You should now be able to use Python 3 to talk to Minecraft.
\end{enumerate}

\textbf{Optional Explanation:} The first command (\textbf{sudo apt-get update})
downloads an up to date list of the packages that are available for
your Operating System. The second command (\textbf{sudo apt-get
  upgrade}) selects all of the currently installed packages which can 
be upgraded, downloads and upgrades them. The third command
(\textbf{sudo apt-get dist-upgrade}) checks if anything major has been 
done to upgrade the operating system as a whole, downloads and
installs them (this will make it possible for you to use Python 3 to
talk to Minecraft rather than Python 2).

\section*{Choosing not to update}

As a minimum you will need a Raspberry Pi which has Minecraft (for the
Raspberry Pi) installed. This will automatically put the Python files
you need in the correct place for Python to find them. If you manually
installed Minecraft you may not have the correct files in the correct
places and it is recommended that you update.

Older versions of Minecraft came with API files (the special
files which tell Python how to talk to Minecraft) which only worked
with Python 2. Which is fine but has slightly different syntax to
Python 3 that most schools are using when teaching Python.
If you choose not to update, then you may find that programs are called
by different names than those described in these documents, or are
located in different places in the Start Menu. It is highly
recommended that you use as up to date a version of the Operating
System as possible but sometimes this is not feasible in a School
environment. 

\section*{Putting a new image onto the SD Cards}
The Raspberry Pi website always contains up to date images. Go to
{\textcolor{greenish} {\url{https://www.raspberrypi.org/downloads/}}} and download either the
NOOBS image or the Raspbian image. \vspace{0.5cm}

If you downloaded the NOOBS image use the guide here:
{\textcolor{greenish} {\url{https://www.raspberrypi.org/help/noobs-setup/}}} to help you use it
to install Raspbian. \vspace{0.5cm}

If you downloaded the Raspbian image directly then use the guide here:
{\textcolor{greenish} {\url{https://www.raspberrypi.org/documentation/installation/installing-images/README.md}}}
to guide you through the process of installing it onto your SD Card. \vspace{0.5cm}

If you bought a card with a pre-installed image on it, then you should
attempt to update it as per the instructions above to ensure that it
contains the latest versions of files.

\section*{Which version of Python?}
\begin{enumerate}
\item Find Python 3 in the Start Menu or on the Desktop, depending on how up
to date your Operating System is this might be called Python3 or
IDLE3. 
\item Open it up. This should display the Python Shell, an
interactive interpreter which allows you to communicate with
Python. 
\item Double check that you have opened the Python 3 version by
looking at the first line in the window, it should mention the Python
version (something like Python 3.2.3 and then a date). 
\item You should also see three arrows: \verb|>>>| This is where we can type
  our commands. 
\item At this command prompt type the following: 
\lstinline{import mcpi.minecraft} and hit enter.
\item If you just see a new line with three more arrows then you are
  good to go and can use Python 3 to talk to Minecraft. If this is the
  case you can ignore the rest of this section.
\item If instead you see some red writing, double check your typing
  and if everything was correctly typed but still gives you red
  writing then it is likely that the required files are not available
  for Python 3.
\item If the Python 3 files are not installed you will have to use
  Python 2 instead. Close the Python 3 shell and instead open up the
  Python 2 version which is probably called Python 2, IDLE2 or
  potentially just Python or IDLE.
\item When the shell opens this time it should have the word Python
  followed by 2 followed by another number (probably a 7). This tells
  you that you have started the Python 2 shell.
\item At the command prompt indicated by the three arrows \verb|>>>| again
  type: \lstinline{import mcpi.minecraft} and hit enter.
\item If you see red writing this time then Minecraft is not installed
  at all for either of your versions of Python and you will need to
  either update your Operating System or download and install a new
  version. 
\item If you just see a new version of the command prompt \verb|>>>| with
  no red writing then you are good to go and just need to remember
  that all of your Minecraft Python programs need to be run using
  Python 2.
\end{enumerate}

\section*{Getting the maps in the right place}
The sections entitled On the first go and After the first go, detail
how to use a reasonably automated method to reset the maps for
whichever location you are working with. Unfortunately this process
does not seem to always work properly (especially if you have chosen
to use an older version of the Operating System). Therefore if the
automated methods don't work then there are also details of how to
manually reset the maps. \vspace{0.5cm}

\textbf{USING THE FOLLOWING METHODS WILL OVERWRITE ANY PREVIOUS
  VERSIONS OF THE MINECRAFT MAPS YOU HAVE USED AND ANY CHANGES MADE BY
  STUDENTS WILL BE LOST. SEE THE FINAL SECTION \textit{KEEPING THE WORK} TO
  FIND OUT HOW TO AVOID THIS.}

\section*{On the first go}
\begin{enumerate}
\item Go into the folder you unzipped previously (Double click on its'
  folder icon to do this)
\item In the folder, find the Maps folder, and go into that folder.
\item Find the file called \textbf{ResetGeoCraftMaps.py}.
\item Right click on the file.
\item Select Properties.
\item Click on the Tab marked Permissions.
\item Click on the options to the right of the word Execute.
\item Select the word Anyone from the list.
\item Click Okay.
\item Move on to the next section now.
\end{enumerate}

\section*{After the first go}
\begin{enumerate}
\item Make sure that Minecraft is not running a map at the moment
  (Quit to title if it is).
\item Double click on the file called \textbf{ResetGeoCraftMaps.py}.
\item In the window which pops up select Execute not Execute in terminal.
\item When the new window opens, read the warning and click the button
  if you wish to proceed.
\item This will place a clean version of the map you've downloaded
  into the right place for Minecraft to find it. 
\item It will replace any existing copies of the map with new ones,
  and thus erase any changes you've made to the map in Minecraft.
\end{enumerate}

\section*{Manually updating the maps}
\begin{enumerate}
\item Ensure that Minecraft is not running.
\item You will need to go in to the downloaded folder which you unzipped
(Double click on its' folder icon to do this).
\item Find the Maps folder and go in to that by double clicking its'
  icon.
\item Highlight the folder or folders which actually contain the map, these will
  normally be called by some version of the name of the place. 
\item Select copy from the Edit Menu to copy the folders you have
  highlighted.
\item Go to the home folder for the Raspberry Pi (if you can see the
  folder list on the left it will have a picture of a house with the
  name pi next to it click on it, or click on the image of house on
  the menu bar if you can see it, or in the text box in the menu bar
  type /home/pi and hit Enter)
\item Once you are in the Home folder you need to go to the View Menu
  and select Show Hidden to show all of the hidden files and folders
  in there
\item You need to double click the folder marked .minecraft (notice
  the full stop at the beginning) to enter it.
\item Double click the games folder
\item Double click the com.mojang folder
\item Double click the minecraftWorlds folder
\item Delete any pre-existing Minecraft maps folders in this folder
  which you are not wishing to keep (such as folders with the same
  names as the ones you are about to paste). If you want to keep the
  maps then rename the existing folders before you paste them (TIP if they
  contain a specific student's work rename the folder to have their
  name). 
\item Paste the map folders you copied earlier (CTRL-V, or right click
  and paste or select Paste from the Edit Menu).
\item Start Minecraft, start a game, to select a world you can
  navigate left and right using the A and D keys, to start the world
  you have selected click on it with the mouse. \textbf{DO NOT CLICK
    ON START NEW} as this ignores the world you have selected and just
  creates a new one.
\end{enumerate}

\section*{Keeping the work}
If you need to keep a specific Map/World in which a student has been
working, then you need to follow Steps 6-12 in the section entitled
Manually updating the maps.

\end{document}

%%% Local Variables: 
%%% coding: utf-8
%%% mode: latex
%%% TeX-master: t
%%% End: 
