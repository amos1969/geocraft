
% festival-of-light.tex - Our first LaTeX example!

\documentclass{geocraft-worksheet}

\usepackage[pdftex,
            pdfauthor={Sarah Zaman \& Dave Ames},
            pdftitle={GeoCraft: Wear It! Part 1},
            pdfcreator={LaTeX with hyperref and listings}]{hyperref}

\begin{document}

\title{Wear It!}
\subtitle{Part 1}

\date{}

\maketitle

\pagenumbering{gobble}

\section{Make a Path of Blocks}
\lstset{language=Python}

\noindent% Line above MUST be blank
\tikzmark{Start}%
\lstinputlisting{trails1.py}\vspace{0.2cm}
\tikzmark{End}
\AddBackgroundImage{Start}{End}{mosi}%
%

\noindent This will make a path of stone blocks appear under your feet wherever you go.\vspace{-0.3cm}
\begin{itemize}
\item\textbf{Bronze Challenge:} Change the \textbf{x, y}, and \textbf {z} values so the blocks appear in different
  places near to you.   
\item\textbf{Silver Challenge:} Try using different block types (see the sheet for acceptable block types).
\item\textbf{Gold Challenge:} By adding two lines of code and changing a third we can make a trail of randomly coloured
  wool blocks follow us wherever we move. 
\begin{enumerate}
\item Add a new line after the first two, which reads \lstinline{import random}
\item Add a new line after the \lstinline{while True:} line which reads \lstinline{colour = random.randint(0, 15)} 
\item Modify the last line of code to read \lstinline{mc.setBlock(pos.x, pos.y-1, pos.z, block.WOOL.id, colour)} 
\end{enumerate}
\end{itemize}

\section{Make the CodeBug Do Stuff}

\noindent% Line above MUST be blank
\tikzmark{Start}%
\lstinputlisting{codebug-part1.py}\vspace{0.2cm}
\tikzmark{End}
\AddBackgroundImage{Start}{End}{mosi}%
%

\noindent This makes a message scroll across the LEDs on the CodeBug

\begin{itemize}
\item\textbf{Bronze Challenge:} What happens if you remove the minus sign before the i? Why?

\item\textbf{Silver Challenge:} Change the contents of the message. What other value needs changing?

\item\textbf{Gold Challenge:} By adding a line that reads \lstinline{while True:} before the 
        \lstinline{for i in range(50):} line, and indenting the last 3 lines another time, make the message repeat. 

\end{itemize}

\end{document}

%%% Local Variables: 
%%% coding: utf-8
%%% mode: latex
%%% TeX-master: t
%%% End: 
